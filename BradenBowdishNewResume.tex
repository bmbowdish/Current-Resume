%%%%%%%%%%%%%%%%%%%%%%%%%%%%%%%%%%%%%%%%%
% Medium Length Professional CV
% LaTeX Template
% Version 2.0 (8/5/13)
%
% This template has been downloaded from:
% http://www.LaTeXTemplates.com
%
% Original author:
% Trey Hunner (http://www.treyhunner.com/)
%
% Important note:
% This template requires the resume.cls file to be in the same directory as the
% .tex file. The resume.cls file provides the resume style used for structuring the
% document.
%
%%%%%%%%%%%%%%%%%%%%%%%%%%%%%%%%%%%%%%%%%

%----------------------------------------------------------------------------------------
%	PACKAGES AND OTHER DOCUMENT CONFIGURATIONS
%----------------------------------------------------------------------------------------

\documentclass{resume} % Use the custom resume.cls style

\usepackage[left=0.75in,top=0.6in,right=0.75in,bottom=0.6in]{geometry} % Document margins
\newcommand{\tab}[1]{\hspace{.2667\textwidth}\rlap{#1}}
\newcommand{\itab}[1]{\hspace{0em}\rlap{#1}}
\name{Braden Bowdish} % Your name
%\address{123 Pleasant Lane \\ City, State 12345} % Your secondary addess (optional)
\address{(484)-680-5549 \\ braden@bowdish.me} % Your phone number and email
\address{github.com/bmbowdish}

\begin{document}


%----------------------------------------------------------------------------------------
%	WORK EXPERIENCE SECTION
%----------------------------------------------------------------------------------------

\begin{rSection}{Experience}
\begin{rSubsection}{Geisel Software, Inc.}{February 2019 - September 2019}{Software Intern}{}
\item Collaborated on iOS app for Class 3 Medical Device
\item Worked with CoreBluetooth framework extensively
\item Used virtualization tools like Vagrant, AWS, and VirtualBox
\end{rSubsection}

\begin{rSubsection}{Real Random}{November 2018 - July 2019}{iOS App Developer}{}
\item Designed and developed iOS app in Swift
\item Two-factor authentication app with QR Reader
\item First app on the App Store
\end{rSubsection}

\begin{rSubsection}{Penn Mutual Life Insurance Company}{June 2018 - December 2018}{DevOps Intern}{}
\item Wrote Bash and Python scripts used for Linux administration
\item Worked with Vagrant and Packer to create testing environments
\end{rSubsection}
\end{rSection}

%------------------------------------------------



%	EXAMPLE SECTION
%----------------------------------------------------------------------------------------

\begin{rSection}{Personal Projects} \itemsep -2pt
\begin{rSubsection}{Swiftfall}{Spring 2018}{https://github.com/bmbowdish/Swiftfall}{Swift, JSON, APIs}
\item Open-source wrapper for the Scryfall API
\item Makes it easier to develop MTG apps on iOS
\end{rSubsection}
\begin{rSubsection}{Magic App}{Winter 2018}{https://github.com/bmbowdish/BrickHack-Magic-App}{Swift, iOS}
\item Designed an iOS and Apple Watch app
\item Tool for players of trading card games
\end{rSubsection}
\begin{rSubsection}{Event Keeper}{Fall 2015}{https://github.com/bmbowdish/Event-Keeper}{Python, Arduino, Raspberry Pi, Raspbian}
\item Event Keeper can take attendance during group meetings
\item Python Application running on a Raspberry Pi
\end{rSubsection}
\end{rSection}

%----------------------------------------------------------------------------------------
%	TECHNICAL STRENGTHS SECTION
%----------------------------------------------------------------------------------------

\begin{rSection}{Technical Strengths}

\begin{tabular}{ @{} >{\bfseries}l @{\hspace{6ex}} l }
Programming Languages &  Swift, Python, Bash, C, Java \\
Software \& Technologies & Linux, Git, GitHub, Vagrant, Bluetooth, APIs, Xcode\\
\end{tabular}
\end{rSection}

%----------------------------------------------------------------------------------------
%	EDUCATION SECTION
%----------------------------------------------------------------------------------------

\begin{rSection}{Education}

{\bf Rochester Institute of Technology} \hfill {\em 2015 - 2018} 
\\ Computer Science (Incomplete) \\
Member of Computer Science House \\

\end{rSection}
%----------------------------------------------------------------------------------------
\end{document}

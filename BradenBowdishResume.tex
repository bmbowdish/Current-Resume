% !TXS template
\documentclass[10pt,a4paper,sans,english]{moderncv}        % possible options include font size ('10pt', '11pt' and '12pt'), paper size ('a4paper', 'letterpaper', 'a5paper', 'legalpaper', 'executivepaper' and 'landscape') and font family ('sans' and 'roman')
\moderncvstyle{classic}                             % style options are 'casual' (default), 'classic', 'oldstyle' and 'banking'
\moderncvcolor{blue}                               % color options 'blue' (default), 'orange', 'green', 'red', 'purple', 'grey' and 'black'
%\nopagenumbers{}                                  % uncomment to suppress automatic page numbering for CVs longer than one page
\usepackage[utf8]{inputenc}                       % if you are not using xelatex ou lualatex, replace by the encoding you are using
\usepackage[scale=0.7,a4paper]{geometry}
\usepackage{babel}
%----------------------------------------------------------------------------------
%            personal data
%----------------------------------------------------------------------------------
\firstname{Braden}
\familyname{Bowdish}
\mobile{484 680 5549}
\email{bmbowdish@live.com}
\extrainfo{Github:  https://github.com/bmbowdish}


\begin{document}
%-----       resume       ---------------------------------------------------------
\makecvtitle
\section{Objective}
 \cvlistitem{Seeking a Software Engineering related Co-Op for Summer and/or Fall of 2018 and Spring and/or Summer of 2019.}
\section{Education}
  \cvlistitem{Rochester Institute of Technology. Bachelors of Science, Computer Science.}
  \subsection{Relevant Classes}
    \cvlistitem{Data Structures}
    \cvlistitem{Object Oriented Programming}
    \cvlistitem{Mechanics of Programming}
    \cvlistitem{Intro to Software Engineering}
\section{Projects}
  \cventry{--}{Magic App}{\href{https://github.com/bmbowdish/BrickHack-Magic-App}}{(github.com/bmbowdish/BrickHack-Magic-App)}{iPhone and Apple Watch app developed and designed by myself. Created for players of Magic the Gathering who desire an app to assist in gameplay.}{Spring 2017}{}
  \cventry{--}{Event Keeper}{\href{https://github.com/bmbowdish/Event-Keeper}{(github.com/bmbowdish/Event-Keeper)}}{Event Keeper is a Python application which controls an Arduino that reads iButtons and is run on a Raspberry Pi. Event Keeper can take attendance during Computer Science House meetings.} {Fall 2015}{}
  \cventry{--}{LED Matrix}{\href{https://github.com/bmbowdish/LED-Hackathon-Matrix}}{(github.com/bmbowdish/LED-Hackathon-Matrix)}{A simple LED Matrix with a joypad controller. In developing the project I learned to solder and understand basic electronics.}{Fall 2016}{}
  \cventry{--}{Web Checkers}{}{Online Checkers game built in java using the Spark framework. Working with four other developers for 15 weeks.}{Spring 2017}{}
\section{Work Experience}
  \cventry{--}{TD Alfredo's Pizzeria}{}{Worked as a Cashier and Delivery Driver in a local Pizza shop}{2016-2018}{}
\section{Technical Skills}
  \cvitemwithcomment{}{C, Python, Java, Git, Github, Swift, \LaTeX}{}
\section{Extracurriculars}
  \cvitem{--}{Computer Science House, a living-learning community for people interested in computers and other forms of technology.}
\end{document}
